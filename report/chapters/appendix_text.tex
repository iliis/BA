\chapter{Usage of Handed-In Data [TODO: title]}
\label{sec:usage}

Do you want doku for the code here? how to compile, configure and run?

Doku für Simulation in Blender.

How to use [sub-]apps written in C++.

Organization and usage guide for Matlab.

how to compile and use C++ code.

\section{Directory Overview}

All figures, LaTeX and LibreOffice files for the presentation and this report
are in the respectively named directories. Other important folders are:

\begin{description}
    \item[matlab]
        Contains the full warping pipeline implemented in Matlab. Slow, but useful for debugging and automatic derivation of the Jacobians.
    \item[matlab/input]
        Contains various example scenes for the simulation.
        The courtyard scene was downloaded from \texttt{http://www.luxrender.net/forum/viewtopic.php?f=14\&t=6015} and can be used with LuxRenderer
    \item[cpp] Implementation of all warping code in C++ and a few helpful GUIs to visualize things (including a client for the vi-sensor application).
    \item[arm] Plumbing to run photometric odometry on vi-sensor. All my code is in the 'src/odometry' subdirectory.
\end{description}


\section{Blender Simulation}

For debugging and testing purposes, Blender was used to render images with
disparity data and perfect ground truth.  There are several scenes in
\texttt{matlab/input}. Upon first opening a scene, the ground truth
recording script must be registered once by running the
\texttt{output\_frame\_pos.py} once (or by unsafely reloading the .blend-file by
clicking on the warning in the Blender header menu).

This script will now run every time an animation is rendered and will output
relative camera poses into a CSV file called
\texttt{camera\_trajectory\_relative.csv} in the same directory. Upon first
execution, it will also generate \texttt{camera\_intrinsics.csv} containing all
important parameters of the virtual camera such as focal length and how to
scale the raw disparity values.

The generated CSV files and the rendered images can be read by both the C++ and
the Matlab implementation.


\section{Matlab Implementation}

All Matlab functions generally expect their input data in column matrices (for
example, a list of 100 3D points would have 3 rows and 100 columns).

Everything directly implementing warping as described in
section~\ref{sec:warping} can be found in \texttt{core}. This warping pipeline
uses Matlab's matrix operations for speed, but analytically derives the
Jacobians for every iteration, which obviously slows down things immensely.

A good starting point are the scripts in \texttt{test\_scripts}, which provide
various examples of how to use the code: \texttt{run\_minimization.m}
