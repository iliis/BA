\chapter{Introduction}
\label{sec:introduction}

\section{Motivation}
\label{sec:motivation}

Robots are generally constrained in their computational power and energy usage.
A powerful method is to offload computation onto an FPGA, which is usually
a much more efficient than a general purpose processor.  However, programming
an FPGA is not straightforward and integtrating it with code running on a CPU
can be tricky.

In this work, a novel example of such an integration is provided by running a
semi-global stereo matching \cite{hirschmuller2005sgm} core developed in
\cite{honegger2014sgmcore} on the FPGA and using its output for a photometric
visual odometry algorithm \cite{comport2007odometry} running on the CPU.

This approach of photometric odometry does not track a sparse set of features,
as is usuallty done in visual odometry, but instead warps the full image to
find a perspective where the warped image matches the previous frame.

This approach is well suited for offloading to an FPGA, as most parts are
highly parallelizable. Note tough, that this is not the most efficient way to do embedded odometry,
as a lot more data has to be processed. The main goal was to explore how an
FPGA and a general purpose processor can be integrated on an embedded device
and to ascertain the potential for further optimizations by transfering more
parts to the FPGA.

A visual-inertial sensor developed by the ASL \cite{nikolic2014synchronized} is
used which features a Xilinx Zynq 7020 SoC consisting of a dual-core ARM Cortex
A9 and an ARTIX-7 FPGA. The sensor has a wide-angle stereo camera with a
resolution of $752 \times 480$ pixels with syncronized global shutters as well
as a high-precision inertial measurement unit, which was not used here. This
vi-sensor is not only small and lightweight ($133 \times 57$mm, \unit[130]{g}),
it is also power-efficient, consuming less than \unit[10]{W}. [TODO: 500mA @ 5V are only 2.5W!]



\section{Related Work}
\label{sec:related_work}

Photometric odometry as initially developed by Comport et al. in
\cite{comport2007odometry} has been implemented in \cite{omaridenseodometry} to
use data from the SGM core but running on a powerful PC instead.

In \cite{marcin2014odometry}, feature-based odometry running on the visensor is
developed, which uses the FPGA for corner detection.



TODO: googlen was es sonst noch so gibt?

TODO: expand that section with a more general overview?
