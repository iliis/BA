\chapter{Method}
\label{sec:method}

Hier kommt die ganze Mathe hin.

explain warping:

we have pixels with depth data, which in other word is a point cloud. I.e. we
can project pixels into 3D space where they can be warped trough time and
space. Then, we reproject them back onto the camera's image plane and get a new
picture which is the old picture viewed from a different perspective ("warped").

Now we can compare this warped image with a previous frame (called 'The Keyframe') by simply calculating the squared error of the pixel intensities. We now have a function e(T) which we can minimize using classical approaches, like Gauss-Newton:

Derive that thing ($ J_I * J_P * J_T $) and use $J^T*J*dT = J^T*e(T)$.

To speed things up (and hopefully to increase convergence radius, but I somehow doubt this a bit), we use an image pyramid: scale down images, find minimum, scale up a bit again, improve transformation estimation, repeat.
