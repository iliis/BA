\chapter{Implementation Details}

A few things that are only tangentially related to the core algorithm but which
nevertheless might be encountered in an actual implementation:

\section{Representation of Transformation}

While translations are mathematically very straightforward, rotations can be
represented in numerous ways (Euler-angles, quaternions, rotation matrices,
etc.) and proper derivation of the Jacobians can be tricky.

Fortunately, odometry works in a relative fashion without any absolute
orientation and steps are very incremental as photometric odometry cannot
handle more than a few degrees of rotation. This implies we do not have to deal
with gimbal-lock and other mathematical hurdles of working in $SO(3)$.

\section{Image Scaling}

Care has to be taken to properly downscale image coordinates when working with
image pyramids. This can be done by scaling the camera intrinsics properly:
Halving the image width also means halving the focal length and doubling the
baseline.

Downscaling usually implies filtering and doing so alters the 3D structure. For
this reason, \cite{comport2007odometry} does not downscale the disparity values and
samples them at the full resolution. Though when matching pose on the camera plane
instead of in 3D space, downscaling the disparity values works as well
\footnote{It might be worth investigating how much of an effect on performance
and quality downscaling the disparity images has. Disparity values are only
read at integer coordinates and downscaling them might not be worth the runtime
penalty.}.

\section{Ignore Invalid Pixels}

Pixels that do not have a disparity value (because the SGM algorithm could not
find any correspondence) can obviously be ignored. So can pixels which are
saturated or underexposed: They provide neither a good gradient nor valid
disparity data and are often not photoconsistent.

\section{Keeping Pixel Numbers Stable}

The optimizations described in section~\ref{sec:optimizations} can
substantially reduce the pixel count, so much so that there are not enough for
stable performance. Especially filtering pixels based on their image gradient
as explained in section~\ref{sec:gradient_filtering} requires a well-chosen
threshold, as image gradients depend on the scene. \cite{comport2007odometry}
computes a histogram to select the $N$ best pixels. An easier approach is to
simply restart the current iteration with a lower threshold when the number of
pixels gets too low and increasing it after a step with enough pixels, as in
\cite{omaridenseodometry}. The same problem also applies to other optimization
parameters such as the size of the image pyramid.
